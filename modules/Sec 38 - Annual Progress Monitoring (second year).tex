\newpage
\chapter{Annual Progress Monitoring (second year) (01/06/2018)}

Just as last year, I had to go through the Annual Progress Monitoring (APM) for my second year. I had to write a report: %\href{run:./sec38/SecondYearReport.pdf}{SecondYearReport.pdf}
 and go through an interview. The report outlined the work I'd done over my second year, including the SUSY \alphat\ analysis, the EXO semi-visible jets analysis, the HIG Higgs to invisible analysis, and service work. I also had to discuss future plans.

Notes from the interview (13/08/2018) to follow up on:

\begin{easylist}
\ListProperties(Style*=-- , FinalMark={)})
& Research the motivations for studying/trying to discover dark matter at the LHC. These include the WIMP miracle and cosmological constraints on the $\sigma \times$BR for dark matter.
& Talk to Henning about a proper thesis outline that include chapter headings and overall details.
& Figure out timelines for semi-visible jets and Higgs to invisible analyses so that I can start making deadlines about when to get things finished by and when to start writing parts of my thesis.
\end{easylist}


\section{Thesis outline}

The following contains my thesis outline written in August 2018:

\begin{easylist}
\ListProperties(Style*=-- , FinalMark={)})
& Dedication
& Acknowledgements
& Abstract
& Contents
\end{easylist}
\

Introduction to High Energy Physics:

\begin{easylist}
\ListProperties(Style*=-- , FinalMark={)})
& Give an overview of particle physics in general. Discuss some of the history of the field, CERN and the LHC.
& Give an overview of the fundamental forces and particles.
& Give an overview of the state of dark matter, evidence for its existence, motivations to study it in general and at the LHC.
\end{easylist}
\

Theory (the Standard Model and Dark Matter):

\begin{easylist}
\ListProperties(Style*=-- , FinalMark={)})
& Discuss the Standard Model in detail, emphasising certain aspects as they relate to dark matter and the Higgs field (and boson).
& Discuss the theory behind the semi-visible jets analysis: strongly interacting dark sector in Hidden Valley scenario with a portal to the visible sector. Mentioning dark quarks, dark confinement scale, dark hadronisation and decay, running coupling, etc.
& Discuss the theory behind combined Higgs to inv.: branching fraction of H -> nunu set at 1\% (include context) whilst current experimental limit is ~24\%
\end{easylist}
\

The Large Hadron Collider and CMS experiment:

\begin{easylist}
\ListProperties(Style*=-- , FinalMark={)})
& Explain CERN and the LHC in more detail.
& Give an overview of the CMS experiment and detector (including all subsystems and object identification).
& Either as a subsection in this chapter or in a separate chapter, discuss the Level-1 Trigger in depth. Emphasise the jet and energy sum triggers as I've worked on them, and Calorimeter Layer-2 for the same reason.
\end{easylist}
\

A search for dark matter from a semi-visible jet final state:

\begin{easylist}
\ListProperties(Style*=-- , FinalMark={)})
& Discuss how the theoretical aspects from the Theory chapter translate into an experimental search.
& Include object definitons, overall analysis strategy, triggers(?), signal production, event selection, background estimation and results/limit (including comparisons to similar searches).
& Current material: no public plots as of yet. Hope to have enough material for a PAS (if required) for Moriond 2019.
\end{easylist}
\

A search for dark matter from constraining the Higgs boson to invisible state decay mode:

\begin{easylist}
\ListProperties(Style*=-- , FinalMark={)})
& Discuss how the theoretical aspects from the Theory chapter translate into an experimental search.
& Include object definitons, overall analysis strategy, triggers(?), signal production, event selection, background estimation and results/limit (including comparisons to previous results).
& Current material: no public plots as of yet. Hope to have enough material for a PAS (if required) for Moriond 2019.
\end{easylist}
\

Summary, Conclusions and Prospects:

\begin{easylist}
\ListProperties(Style*=-- , FinalMark={)})
& Include a summary of thesis and work done over the course of my PhD with emphasis on the most important results/contributions.
& Mention the direction the semi-visible jet and Higgs to invisible analyses can take (sharing ideas/strategies I have, potential improvements with more LHC data and future prospects from potential future experiments).
\end{easylist}
\

\begin{easylist}
\ListProperties(Style*=-- , FinalMark={)})
& Appendices if required
& Index/glossary of HEP terms and acronyms
& Bibliography/references (either at the end of the thesis itself or at the end of each chapter)
\end{easylist}
\

Timeline:

\begin{easylist}
\ListProperties(Style*=-- , FinalMark={)})
& Combined Higgs to Invisible plans: "preliminary analysis" for Moriond 2019. Should consist of a first-pass analysis of 2016-2018 data with all four Higgs to invisible production modes and approved plots.
& Semi-visible jets plans: "preliminary analysis" for Moriond 2019. Should consist of a first-pass analysis of 2016-2017 data with s-channel signal. Will hopefully include t-channel either simultaneously or at a later date.
& Thesis plans: start writing by September 2019 for hand-in in March 2020. By September 2019, both above analyses should be well under way. Ideally, the papers will have been approved/published by this time. If not, they should be close and have produced all necessary plots (that I can use as placeholders until the final versions are ready).
\end{easylist}
\

Omission of 2016 paper on natural and split SUSY:

\begin{easylist}
\ListProperties(Style*=-- , FinalMark={)})
& Right now, the goal is just to include analysis chapters on semi-visible jets and combined Higgs to invisible. Henning suggested omitting natural/long-lived SUSY analysis for now as it makes thesis less coherent.
& Depending on how much progress is made with the two analyses that are currently planned, we may include a short chapter on SUSY.
\end{easylist}
\

