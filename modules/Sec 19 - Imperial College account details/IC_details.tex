
\chapter{Imperial College account details (01/02/2017)}
\label{sec:imperialdeets}

My details for my Imperial guest account are as follows:

\begin{easylist}[itemize]
\easylistprops
& Email address: e.bhal@imperial.ac.uk
& College Identifier (CID): 942250
& Hostnames: lx00.hep.ph.ic.ac.uk $|$ lx01.hep.ph.ic.ac.uk
& Aliases for hosts: imperial00 $|$ imperial01
& Username: ebhal
& Password: $<$normal password$>$
\end{easylist}

I can just ssh into the remote server like I do with Soolin -- \texttt{ssh ebhal@lx0<\#>.hep.ph.ic.ac.uk} or \texttt{ssh imperial0<\#>}. These servers are Linux based and run CentOS6. But I can ssh into the Imperial servers from anywhere in the world on any network (check this), so I don't have to be on a specific network like I do for Soolin. More information can be found at \url{https://www.hep.ph.ic.ac.uk/private/computing/}.


\section{lx00 and lx01 remote server details}

I can ssh into both lx00 and lx01, they both take me to the same place. So I don't need to maintain two setups at IC. They have a grid like Bristol does. However, their batch submission and monitoring software is slightly different. They use Grid Engine (SGE) rather than HTCondor. A rundown of the batch system is located at \url{https://www.hep.ph.ic.ac.uk/private/computing/sge.shtml}. The main takeaways are that the command for job submission is

\begin{lstlisting}[belowskip=-0.7cm, language=sh, numbers=none]
qsub -q hep.q <submission script>
\end{lstlisting}

I can also specify a maximum time for a job by adding the option \texttt{-l h\_rt=<hours>:<minutes>:<seconds>}. The maximum, and also default, is 48 hours. Typing \texttt{man qsub} and \texttt{man sge\_intro} displays the manual and other commands, respectively. I can check on jobs by typing

\begin{lstlisting}[belowskip=-0.7cm, language=sh, numbers=none]
qstat
\end{lstlisting}

I can delete jobs by typing

\begin{lstlisting}[belowskip=-0.7cm, language=sh, numbers=none]
qdel <job ID>
\end{lstlisting}

where \texttt{job ID} is the number in the first column of output when typing \texttt{qstat}. Or I can type

\begin{lstlisting}[belowskip=-0.7cm, language=sh, numbers=none]
qdel -u ebhal
\end{lstlisting}

to kill all my jobs. I can also get more information about a job with

\begin{lstlisting}[belowskip=-0.7cm, language=sh, numbers=none]
qstat -j <job ID>
\end{lstlisting}
