
\chapter{CERN Account (14/10/2016)}
\label{sec:cernaccount}

After some work, I've managed to get my CERN account up and running. Details are below.
\begin{easylist}
\ListProperties(Style*=, FinalMark={)})
& Name: Eshwen Bhal
& Login/Username: ebhal
& Email address: eshwen.bhal@bristol.ac.uk
& CERN email address: eshwen.bhal@cern.ch
& Password (for login, EDH, mail): see Sec.~\ref{sec:introduction} about cipher
& Hypernews User ID: ebhal
& Hypernews password: cernhypernews
\end{easylist}

To get my CERN email account on an android device, I just have to enter the email address and password when prompted. To get it on the Mail app on macOS, I can add it as an Exchange account. Some useful links (which I've also bookmarked in Firefox) are

\begin{easylist}
\ListProperties(Style*=, FinalMark={)})
& Account Management: \url{https://account.cern.ch/account/}
& Email: \url{https://mmm.cern.ch/owa/}
& Android Community: \url{https://android.web.cern.ch/}
\end{easylist}

I've also subscribed to the CMS HyperNews System \sloppy\url{https://hypernews.cern.ch/HyperNews/CMS/top.pl} which will keep me up to date with advances in the field and news related to the forums I'm subscribed to (Exotica and Exotica-MET+X, at the moment). The twiki page at CERN is \sloppy\url{https://twiki.cern.ch/twiki/bin/view/CMS/EXOTICA}. To sign up initially to HyperNews I have to go to the terminal and type \texttt{ssh ebhal@lxplus.cern.ch} to log in to CERN's LXPLUS cluster (their Soolin equivalent), then \texttt{ssh hypernews.cern.ch} and set up a username and password. My username is just \texttt{ebhal} and my password is \texttt{cernhypernews}. Mailing lists that I'm subscribed to can be found on HyperNews and on \sloppy\url{https://e-groups.cern.ch/}. If I end up with a constant barrage of emails, look through my subscriptions and unsubscribe to irrelevant lists. With a CERN account, I also get access to Indico (\sloppy\url{https://indico.cern.ch/}); the service which organises meetings and conferences. A useful link is \url{https://indico.cern.ch/category/201/}, which lists UK CMS meetings (Bristol CMS weekly meetings, RA1, trigger, etc.).


\section{The lxplus remote server}

As a CERN affiliate, I have access to the remote server and grid located on site, lxplus. I can ssh into it with

\begin{lstlisting}[belowskip=-0.7cm, language=sh, numbers=none]
ssh ebhal@lxplus.cern.ch
\end{lstlisting}

As LXPLUS uses the AFS file system, changing permissions of files/folders so that other users can access them is not necessarily as simple as on regular Unix systems (i.e., with \texttt{chmod}). If they don't work, see \url{https://ist.njit.edu/afs-permissions/} for a list of the options.

Disk quotas:

\begin{easylist}
\ListProperties(Style*=, FinalMark={)})
& Home directory \textbf{/afs/cern.ch/user/e/ebhal}: 10 GB
& Work directory \textbf{/afs/cern.ch/work/e/ebhal}: 100 GB
& EOS directory \textbf{/eos/user/e/ebhal}: 1 TB. Can check usage at \url{https://cernbox.cern.ch/}.
\end{easylist}


\section{Certificates and running on the grid(s)}
\label{subsec:gridcertificates}

I've been having the occasional problem accessing CMS/CERN webpages, citing problems with my certificates. If I get this problem in the future, visit \url{https://ca.cern.ch/ca/} and install the following: New Grid User certificate; New EduRoam certificate; CERN Certification Authorities Files; CERN Grid Certification Authorities Certificates. I can also check the expiry dates of the certificates I have by going to "My User certificates".

To use the grid on Soolin or another remote server, I had to go onto Firefox on my Mac, then go to Preferences $\rightarrow$ Advanced $\rightarrow$ Certificates $\rightarrow$ View Certificates $\rightarrow$ Your Certificates, then "Backup" the certificate named "Eshwen Bhal". This should download the certificate to my Downloads folder, with the extension \textbf{.p12}. Rename the file to \textbf{myCert.p12} and \texttt{scp} it to my home directory on the server. Then type the commands

\begin{lstlisting}[belowskip=-0.7cm, language=sh, numbers=none]
openssl pkcs12 -in myCert.p12 -clcerts -nokeys -out $HOME/.globus/usercert.pem
openssl pkcs12 -in myCert.p12 -nocerts -out $HOME/.globus/userkey.pem
chmod go-rw $HOME/.globus/userkey.pem
\end{lstlisting}

to extract my certificate and key, and then initialise the key so that I'm able to start proxies with my certificate. The Import Password is the one I encrypted the certificate with. Then choose the PEM pass phrase as the same one for simplicity. I can then delete the file \textbf{myCert.p12}. These instructions are also available at \url{https://ca.cern.ch/ca/Help/?kbid=024010}. Then I'll be able to issue the command \texttt{voms-proxy-init --voms cms --valid 168:00} to start a proxy of a grid certificate (valid for 168 hours) so that I can run jobs (when prompted, the grid pass phrase is the PEM pass phrase I set earlier). \uline{The certificate itself is only valid for a year, so when I need to renew it I can look at the instructions in the link above to get one. I can just download that to my computer, upload it to each remote server I use and follow the steps above to properly initialise it.}
