
\chapter{Using \texorpdfstring{\PYTHIA}{Pythia} with Cygwin (on Windows) (27/09/2016)}

\begin{easylist}[itemize]
\easylistprops
& Make sure Cygwin is installed (with all the "devel" packages for \texttt{make} commands, etc.).
& Unpack the \PYTHIA tar file in the \textbf{Cygwin/home/} directory.
& Follow the README in the \textbf{pythia8219/} directory to make the programs, etc.
& Follow the example in the \textbf{examples/} folder (using the README in the folder as a guide).
& Generally, just enter into the Cygwin terminal \texttt{make mainNN} (where \texttt{NN} is the two-digit number after the word \texttt{main} in the source files), then type \texttt{./mainNN $>$ output}.
& Then look at the "output" file to get an idea of how the output will generally look when running scripts, etc. I can also look at the source files and the scripts to see what their objectives are.
\end{easylist}

In the command \texttt{./mainNN $>$ output}, the name of the file "output" isn't necessarily fixed. I can change the name of the output file to whatever I want. It just makes it easier to keep track of different simulations/programs I run.

% No point including "timetable for lectures in term 1"
