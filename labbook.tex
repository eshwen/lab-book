\documentclass[12pt, dvipsnames, svgnames]{cmspaper} % dvipsnames, svgnames allows more colours to be used

% In TeXShop, compile pdflatex and bibliography in one command by selecting "pdflatexmk" from the drop-down menu next to "Typeset".
% If compilation with pdflatexmk fails, hit the "Trash Aux Files" button and recompile.

\usepackage{palatino} % this is the font of the main text
\usepackage{ptdr-definitions} % CMS definitions and styles. Needs to come first so definitions from other packages don't cause conflicts

\usepackage{mathtools} % includes the "amsmath" package and other mathematical symbols and environments
\usepackage{microtype} % for font and line spacing. fewer overfull/underfull hbox warnings
\usepackage{texnames}
\usepackage[utf8]{inputenc} % for special/accented characters
\usepackage{physics} % for physics symbols, such as differential symbols
\usepackage{courier} % font for code
\usepackage{float} % to fix figures in place
\usepackage{graphicx} % so I can use \includegraphics, etc.
\usepackage{cprotect} % so I can use \verb and other commands within captions
\usepackage{amssymb} % for an extended library of mathematical symbols
\usepackage{bbold} % for mathematical symbols like the characters for special sets
\usepackage[ampersand]{easylist} % for bullet points and lists, etc. New entry starts with & character
\usepackage{cancel} % for strike-throughs for missing energy/momentum, etc.
\usepackage{maybemath} % required for hepparticles
\usepackage{hepparticles} % required for hepnicenames
\usepackage{hepnicenames} % standard symbols for HEP particles
\usepackage{wasysym} % for more mathematical and astronomical symbols
\usepackage{pdfpages} % to insert PDF documents within this document
\usepackage{spverbatim} % for linebreaks in the verbatim environment
\usepackage{tabularx} % for multii-line cells in tables
\usepackage{multirow} % for nested rows in tables
\usepackage{siunitx} % for aligning numbers properly in tables (column type is "S")
\usepackage[normalem]{ulem} % for underlining over multiple lines. Need "normalem" option otherwise \emph underlines
\usepackage{pdflscape} % for landscape-oriented tables
\usepackage{xfrac} % for nice-looking diagonal (a/b) fractions

\usepackage{array} % for more flexibility with tables
\newcolumntype{M}[1]{>{\centering\arraybackslash}m{#1}} % for text in table cells to be centred

%\usepackage[backend=biber, natbib=true, sorting=none, style=numeric-comp]{biblatex} % old bibliography style, using biber backend/engine
%\addbibresource{mybib.bib} % is using biblatex
\usepackage[square, numbers, sort&compress]{natbib} % new bibliography style using bibtex backend/engine

\usepackage{color} % for more colours
\usepackage{xcolor} % for even more colours (see https://en.wikibooks.org/wiki/LaTeX/Colors for more info)
\definecolor{mygreen}{rgb}{0,0.6,0}
\definecolor{mygray}{rgb}{0.5,0.5,0.5}
\definecolor{mymauve}{rgb}{0.58,0,0.82}

\usepackage{hyperref} % to hyperlink references, contents, etc.
\hypersetup{
    colorlinks=true,
    linkcolor=blue,
    filecolor=Fuchsia,
    urlcolor=red,
    citecolor=cyan,
}

\usepackage{listings} % for displaying code
\lstset{ %
  backgroundcolor=\color{white},   % choose the background color
  % frame=single,		% adds a border around the code
  basicstyle=\normalsize\ttfamily,        % the size of the fonts that are used for the code
  breakatwhitespace=false,         % sets if automatic breaks should only happen at whitespace
  breaklines=true,                 % sets automatic line breaking
  captionpos=b,                    % sets the caption-position to bottom
  commentstyle=\color{mygreen},    % comment style
  deletekeywords={...},            % if you want to delete keywords from the given language
  escapeinside={\%*}{*)},          % if you want to add LaTeX within your code
  extendedchars=true,              % lets you use non-ASCII characters; for 8-bits encodings only, does not work with UTF-8
  keepspaces=true,                 % keeps spaces in text, useful for keeping indentation of code (possibly needs columns=flexible)
  keywordstyle=\color{blue},       % keyword style
  % language=C++,                 % the language of the code
  otherkeywords={*,...},           % if you want to add more keywords to the set
  numbers=left,                    % where to put the line-numbers; possible values are (none, left, right)
  numbersep=5pt,                   % how far the line-numbers are from the code
  numberstyle=\tiny\color{mygray}, % the style that is used for the line-numbers
  rulecolor=\color{black},         % if not set, the frame-color may be changed on line-breaks within not-black text (e.g. comments (green here))
  showspaces=false,                % show spaces everywhere adding particular underscores; it overrides 'showstringspaces'
  showstringspaces=false,          % underline spaces within strings only
  showtabs=false,                  % show tabs within strings adding particular underscores
  stepnumber=2,                    % the step between two line-numbers. If it's 1, each line will be numbered
  stringstyle=\color{mymauve},     % string literal style
  % identifierstyle=\color{orange},	% identifier style
  literate={~} {$\sim$}{1},		% sets the ~ character as it looks in code, not in normal text 
  tabsize=2,	                   % sets default tabsize to 2 spaces
  title=\lstname                   % show the filename of files included with \lstinputlisting; also try caption instead of title
}

\usepackage{tocloft} % for table of contents, figures, tables
\renewcommand{\listfigurename}{Figures} % rename List of Figures heading
\renewcommand{\listtablename}{Tables} % rename List of Tables heading
\renewcommand{\lstlistlistingname}{Source code and scripts} % rename List of Listings heading
\renewcommand\cftsecafterpnum{\vskip10pt} % change line spacing after each section entry
\renewcommand\cftfigafterpnum{\vskip10pt} % change line spacing after each Figure entry
\renewcommand\cfttabafterpnum{\vskip10pt} % change line spacing after each Table entry
\renewcommand\cftbeforesubsecskip{-5pt} % reduce line spacing between section and subsection entry
\renewcommand\cftsubsecafterpnum{\vskip10pt} % change line spacing after each subsection entry
\renewcommand\cftbeforesubsubsecskip{-5pt} % reduce line spacing between subsection and subsubsection entry
\renewcommand\cftsubsubsecafterpnum{\vskip10pt} % change line spacing after each subsubsection entry

\renewcommand{\arraystretch}{1.2} % increase vertical space between rows in a table

\begin{large} % these allow my listings to be added like list of tables/figures (with the same appearance)
\makeatletter
\begingroup\let\newcounter\@gobble\let\setcounter\@gobbletwo
  \globaldefs\@ne \let\c@loldepth\@ne \let\l@lstlisting\l@figure
  \newlistof{listings}{lol}{\lstlistlistingname}
\endgroup
\let\l@lstlisting\l@listings
\AtBeginDocument{\addtocontents{lol}{\protect\addvspace{15pt}}}
\makeatother
\renewcommand{\lstlistoflistings}{\listoflistings}
\end{large}

% Replace existing commands/make my own commands
\renewcommand{\_}{\texttt{\char`_}}
\newcommand{\madgraph}{\MADGRAPH}
\newcommand{\madanalysis}{\textsc{MadAnalysis}\xspace}
\newcommand{\rivet}{\textsc{Rivet}}
\newcommand{\etmiss}{\MET}
\newcommand{\met}{\MET}
\newcommand{\htmiss}{\mht}
\newcommand{\alphat}{\ensuremath{\alpha_{\mathrm{T}}}\xspace}
\newcommand{\biasedDPhi}{\ensuremath{\Delta\phi^*_{\mathrm{min}}}\xspace}
\newcommand{\pT}{\pt}
\newcommand{\LSP}{\ensuremath{\tilde{\chi}_1^0}\xspace}
\newcommand{\rinv}{\ensuremath{r_{\mathrm{inv}}}\xspace}
\newcommand{\comruntwo}{\ensuremath{\sqrt{s} = 13 \TeV}\xspace}

\begin{document}

\begin{titlepage}

   \cmsnote{Lab book}
   \date{\today}

  \title{Dark Matter searches at CMS at \comruntwo}

  \begin{Authlist}
    Eshwen Bhal
       \Instfoot{bri}{University of Bristol, United Kingdom}
       Contact info: \textbf{eshwen.bhal@bristol.ac.uk}; \textbf{eshwen.bhal@cern.ch}
       
       Supervisors: Henning Fl\"{a}cher (\textbf{Henning.Flacher@cern.ch}, \textbf{Henning.Flaecher@bristol.ac.uk}); Bj\"{o}rn Penning (\textbf{penning@cern.ch})
  \end{Authlist}

  \begin{abstract}
This document is an electronic copy of the lab book I'll be keeping during my PhD. It was initially intended both as a backup in case something happens to my physical lab book, and as a searchable document if I want to find a reference, etc., quickly. But now it has assumed the role of my primary lab book for various reasons. This contains everything I have worked on since the start of my PhD including all analysis, research, service work, and also code, commands and reminders of things I may forget. It is also date-logged (in part), fully searchable and hyper-referenced.
  \end{abstract} 
  
\end{titlepage}

\pagenumbering{roman} % use Roman numerals in contents pages
\tableofcontents % adds a contents page
\listoffigures % includes a list of figures in contents
\listoftables % includes a list of tables in contents
\lstlistoflistings % includes a list of listings (e.g. code)

\newpage
\pagenumbering{arabic} % change page numbering back to Arabic numbers

% Include all modules for lab book. Remember that path names to figures and other files will be relative to this directory of *this* file, not the directory where the module is located. I can also comment out the other modules If I want to test the current stuff I'm writing, just so it compiles quicker.

\input{./modules/"Sec 1 - Introduction.tex"}
\input{./modules/"Sec 2 - Introduction, software hardware needs, stuff to remember.tex"}
\input{./modules/"Sec 3 - Useful introductory papers for Supersymmetry and dark matter.tex"}
\input{./modules/"Sec 4 - Using Pythia with Cygwin.tex"}
\input{./modules/"Sec 5 - Interesting thoughts and interpretations of dark matter.tex"}
\input{./modules/"Sec 6 - Running ROOT in a Linux VM.tex"}
\input{./modules/"Sec 7 - Using Soolin and DICE.tex"}
\input{./modules/"Sec 8 - Using MadGraph first steps.tex"}
\input{./modules/"Sec 9 - CERN account.tex"}
\input{./modules/"Sec 10 - Scripting for MadGraph and ROOT.tex"}
\input{./modules/"Sec 11 - Filling, Comparing and Normalizing Histograms.tex"}
\input{./modules/"Sec 12 - CMSSW and Fireworks.tex"}
\input{./modules/"Sec 13 - Using MadAnalysis and searching for dark matter.tex"}
\input{./modules/"Sec 14 - Concepts, nomenclature and definitions.tex"}
\input{./modules/"Sec 15 - Evaluating backgrounds in a dark matter search.tex"}
\input{./modules/"Sec 16 - Rivet.tex"}
\input{./modules/"Sec 17 - Searching for dark matter by identifying invisible jets.tex"}
\input{./modules/"Sec 18 - Cut flow tables for SUS-15-005.tex"}
\input{./modules/"Sec 19 - Imperial College account details.tex"}
\input{./modules/"Sec 20 - Service Work Jet Energy Corrections.tex"}
\input{./modules/"Sec 21 - Using SSHFS to mount remote servers.tex"}
\input{./modules/"Sec 22 - Physics PGR conference and Annual Progress Monitoring.tex"}
\input{./modules/"Sec 23 - Signal model analyses for SUS-16-038.tex"}
\input{./modules/"Sec 24 - Making DMSimp dark matter trees.tex"}
\input{./modules/"Sec 25 - LTA at CERN.tex"}
\input{./modules/"Sec 26 - Using HEPData to archive published material.tex"}
\input{./modules/"Sec 27 - Checking HTmiss tails for data in the event display.tex"}
\input{./modules/"Sec 28 - Using NumPy and root_numpy.tex"}
\input{./modules/"Sec 29 - Tree production for common analysis and 2017 data.tex"}
\input{./modules/"Sec 30 - Cut flow tables for SUS-16-038.tex"}
\input{./modules/"Sec 31 - Analysing the T2tt-4bd SUSY model.tex"}
\input{./modules/"Sec 32 - An introduction to the dark Higgs.tex"}
\input{./modules/"Sec 33 - Service Work Trigger Shifts at P5.tex"}
\input{./modules/"Sec 34 - Service Work - ETmiss studies.tex"}
\input{./modules/"Sec 35 - Semi-visible jets analysis.tex"}
\input{./modules/"Sec 36 - FAST-RA1.tex"}
\input{./modules/"Sec 37 - Combined Higgs to invisible analysis.tex"}
\input{./modules/"Sec 38 - Annual Progress Monitoring (second year).tex"}

%------------------------------------------------------------------------------------------------------------------------

\newpage
\phantomsection % otherwise hyperlink for bibliography goes to the section/subsection before it
\addcontentsline{toc}{section}{References} % to add bibliography to contents. Remove URLs from entries if present.

%\printbibliography % if using biblatex package, this is how the bibliography is printed
\bibliographystyle{unsrt}
\refstepcounter{section}
\bibliography{mybib.bib}
\end{document}
